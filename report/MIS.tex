\documentclass[12pt]{article}

\usepackage{graphicx}
\usepackage{paralist}
\usepackage{amsfonts}
\usepackage{amsmath}
\usepackage{hhline}
\usepackage{booktabs}
\usepackage{multirow}
\usepackage{multicol}
\usepackage{url}

\oddsidemargin -10mm
\evensidemargin -10mm
\textwidth 160mm
\textheight 200mm
\renewcommand\baselinestretch{1.0}

\pagestyle {plain}
\pagenumbering{arabic}

\newcounter{stepnum}

%% Comments

\usepackage{color}

\newif\ifcomments\commentstrue

\ifcomments
\newcommand{\authornote}[3]{\textcolor{#1}{[#3 ---#2]}}
\newcommand{\todo}[1]{\textcolor{red}{[TODO: #1]}}
\else
\newcommand{\authornote}[3]{}
\newcommand{\todo}[1]{}
\fi

\newcommand{\wss}[1]{\authornote{blue}{SS}{#1}}

\title{Game of life, Specification}
\author{Behzad Khamneli}
\date{}
\begin {document}

\maketitle
This Module Interface Specification (MIS) document contains modules, types and
methods for implementing the state of a game of Game Of Life.

\bibliographystyle{plain}
\bibliography{SmithCollectedRefs}

\newpage

\section* {GameBoard ADT Module}

\subsection*{Template Module}

Game

\subsection* {Uses}

N/A

\subsection* {Syntax}

\subsubsection* {Exported Access Programs}

\begin{tabular}{| l | l | l | l |}
\hline
\textbf{Routine name} & \textbf{In} & \textbf{Out} & \textbf{Exceptions}\\
\hline
new Game  & string & Game & out\_of\_range, invalid\_argument\\
\hline
status &&&\\
\hline
getCoordinates &  & seq of (seq of string) & \\
\hline
setAlive & $\mathbb{N}, \mathbb{N}$ & & out\_of\_range \\
\hline
setDead & $\mathbb{N}, \mathbb{N}$ & & out\_of\_range \\
\hline

\end{tabular}

\subsection* {Semantics}

\subsubsection* {Environmental Variable}

fileName: Input file with alive cells.\\ 

\subsubsection* {State Variables}

coordinates: seq of (seq of string)

\subsubsection* {State Invariant}

$|coordinates|$ = 15x15

\subsubsection* {Assumption $\And$ Design Decisions}

\begin{itemize}
    \item The format of the input file should be 'x,y' on each line. x represents row and y represents column.
    \item The top left cell is 0,0. Row starts from 0 and ends at 14. Column starts from 0 and ends at 14.
    \item The Game constructor is called before any other access routine is called on that instance. Once a Game has been created, the constructor will not be called on it again.
    \item For better scalability, this module is specified as an Abstract Data Type (ADT) instead of an Abstract Object. This would allow multiple games to be created and tracked at once by a client.
    \item The getter function is provided, through violating the property of being essential, to give a would-be view function easily(output) and integrated with a game system in the future.
    \item Size of the board is 15x15.
\end{itemize}

\subsubsection* {Access Routine Semantics}
Game(FileName):
\begin{itemize}
    \item transition: Read the points from the file and use those points as index of the array. The input file has location of alive cells in the following format (on each line) : x,y \\x represents row and y represents column.\\
    coordinates := new seq of (seq of string) such that for each line in the input file:\\
(Row\_col\_range(x,y) $\Rightarrow$ coordinates[x][y] = "yes")\\
    \item exception: exc := (File does not exist $\Rightarrow$ invalid\_argument $|$ not\_row\_col(x,y) $\Rightarrow$ out\_of\_range)
\end{itemize}
status():
\begin{itemize}
    \item transition : coordinates := new seq of (seq of string) coordinates such that $(\forall x, y : \mathbb{N} | Row\_col(x,y) : (count\_cells(x,y) < 2 \Rightarrow coordinates[x][y] = "dead" \,| count\_cells(x,y) = 3 \Rightarrow coordinates[x][y] = "yes" \,| count\_cells(x,y) > 3 \Rightarrow coordinates[x][y] = "dead"))$
\end{itemize}
getCoordinates():
\begin{itemize}
    \item output: out := coordinates
    \item exception : none
\end{itemize}
setAlive(x,y):
\begin{itemize}
    \item transition: coordinates[x][y] := "yes" 
    \item exception: exc := (not\_row\_col(x, y) $\Rightarrow$ out\_of\_range)
\end{itemize}
setDead(x,y):
\begin{itemize}
    \item transition: coordinates[x][y] := "dead"
    \item exception: exc := (not\_row\_col(x, y) $\Rightarrow$ out\_of\_range)
\end{itemize}
\newpage

\section* {Local Types}

\subsection*{Local Functions}

Row\_col: $\mathbb{N} \times \mathbb{N} \rightarrow B$\\
Row\_col(row,col) $\equiv$ $(row \ge 0 \land row < 15 \land col \ge 0 \land col < 15)$\\ \\
not\_row\_col: $\mathbb{N} \times \mathbb{N} \rightarrow B$\\
not\_row\_col(row,col) $\equiv$ $(row < 0 \land row \ge 15 \land col < 0 \land col \ge 15)$\\ \\
count\_cells: $\mathbb{N} \times \mathbb{N} \rightarrow \mathbb{N}$\\
count\_cells(row,col) = count such that $(+count:\mathbb{N} | (\exists i,j : \mathbb{N} | i \in [row - 1 ... row + 1] \land j \in [col - 1 ... col + 1]: \lnot (i = 0 \land j = 0) \land copy[i][j] = "yes" \land Row\_col(row+i, col+j)):1)$\\
copy $\equiv$ new seq of (seq of string) COPY such that $(\forall i,j : \mathbb{N} | Row\_col(i,j) : COPY[i][j] = coordinates[i][j])$


\newpage
\section* {Display Module}

\subsection*{Template Module}

Show

\subsection*{Uses}

N/A

\subsection* {Syntax}

\subsubsection* {Exported Access Programs}

\begin{tabular}{| l | l | l | l |}
\hline
\textbf{Routine name} & \textbf{In} & \textbf{Out} & \textbf{Exceptions}\\
\hline
output & seq of (seq of string) &&\\
\hline
printer & seq of (seq of string) && \\
\hline
\end{tabular}
\subsubsection* {Semantics}

\subsubsection* {Environmental Variable}

output: output file with alive cells.

\subsubsection* {State Variables}

None

\subsubsection* {State Invariant}

None

\subsubsection* {Assumptions $\And$ Design Decisions}

\begin{itemize}
    \item Both output and printer functions take a double pointer to a string(coordinates) as a parameter.
\end{itemize}

\subsubsection* {Access Routine Semantics}
output(seq of (seq of string)):
\begin{itemize}
    \item output: Creates a new file "output.txt". Output file has coordinates of alive cells with the following format(on each line): x,y\\
    x represents row and y represents column.
    \item exception: none
\end{itemize}
printer(seq of (seq of string)):
\begin{itemize}
    \item output: $(\forall i,j : \mathbb{N} | i \ge 0 \land i < 15 \land j \ge 0 \land j < 15 : (coordinates[i][j] = "yes" \Rightarrow print "0") \land (coordinates[i][j] = "dead" \Rightarrow print "."))$
    \item exception: none
\end{itemize}

\end {document}